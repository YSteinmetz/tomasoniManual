\thispagestyle{fancy}

\vspace*{40 pt}

\section{\large{DESCRIÇÃO DOS PROCEDIMENTOS PARA OPERAÇÃO DA MÁQUINA}}

Este equipamento é uma máquina avançada projetada para impressão de alta qualidade em caixas de papelão.
Abaixo segue a descrição de procedimentos importantes para a operação da máquina, incluindo as etapas de configuração e operação.

\subsection{Procedimento para ajuste da máquina sem receita}

Atenção, ao ajustar a máquina sem receita não será possível salvar as configurações, para salvar as configurações é necessário ter uma receita carregada.
Não recomendamos ajustar a máquina sem receita, pois não será possível salvar as configurações e os seus ajustes finos serão perdidos ao carregar uma nova receita.

%cria uma nova lista de itens com o nome procedureAdjustmentNoRecipe
\newlist {procedureAdjustmentNoRecipe}{itemize}{1}
%define as propriedades da lista
\setlist [procedureAdjustmentNoRecipe]{label={}, nosep, before=\vspace{10pt}, after=\vspace{10pt}}

%inicia a lista de itens
\begin{procedureAdjustmentNoRecipe}
  \item[\ding{\dingNumber}] Ajustar o carro esquadrejador e as guias de acordo com o tamanho da chapa; Ver: \ref{ihmAlimentacaoTelaAjustesPagina1} - \nameref{ihmAlimentacaoTelaAjustesPagina1};
  \item[\ding{\dingNumber}] Colocar clichê e tintas nas impressoras que serão usadas;
  \item[\ding{\dingNumber}] Colocar o ferramental na máquina;
  \item[\ding{\dingNumber}] Habilitar as impressoras que serão usadas; Ver: 
  \ifmachineType
  \ref{telaComandoImpressoraHabilitaImpressora} - \nameref{telaComandoImpressoraHabilitaImpressora};
  \else
  \ref{sec:comandosImpressoraHabilitaImpressora} - \nameref{sec:comandosImpressoraHabilitaImpressora};
  \fi
  \item[\ding{\dingNumber}] Fazer os ajustes de pressão no anilox conforme a expessura do clichê; Ver: 
  \ifmachineType
  \ref{ajustaPressaoRoloAnilox} - \nameref{ajustaPressaoRoloAnilox};
  \else
  \ref{sec:telaAjustesImpressoraPressaoRoloAnilox} - \nameref{sec:telaAjustesImpressoraPressaoRoloAnilox};
  \fi
  \item[\ding{\dingNumber}] Ajustar a pressão da caixa de vácuo considerando a espessura do clichê e a espessura da chapa; Ver: 
  \ifmachineType
  \ref{ajustaPressaoCaixaVacuo} - \nameref{ajustaPressaoCaixaVacuo};
  \else
  \ref{sec:telaAjustesImpressoraPressaoCaixaVaco} - \nameref{sec:telaAjustesImpressoraPressaoCaixaVaco};
  \fi
  \item[\ding{\dingNumber}] Realizar os outros ajustes de pressão de acordo com a espessura da chapa; Ver: 
    %Criar nova lista com nome pressureAdjustment
    \newlist {pressureAdjustment}{itemize}{1}
    %define as propriedades da lista
    \setlist [pressureAdjustment]{label={}, nosep, before=\vspace{10pt}, after=\vspace{10pt}}
    %inicia a lista de itens

    \begin{pressureAdjustment}
      \ifmachineType

      \item[\ding{\dingNumber}] \ref{telaAjustesAlimentacaoPressaoRoloPuxador} - \nameref{telaAjustesAlimentacaoPressaoRoloPuxador};
      \item[\ding{\dingNumber}] \ref{telaAjustesSlotterAjustePressaoDoAmassador} - \nameref{telaAjustesSlotterAjustePressaoDoAmassador};
      \item[\ding{\dingNumber}] \ref{telaAjustesSlotterAjustePressaoDoEntalheTraseiro} - \nameref{telaAjustesSlotterAjustePressaoDoEntalheTraseiro};
      \item[\ding{\dingNumber}] \ref{telaAjustesSlotterAjustePressaoDoEntalheDianteiro} - \nameref{telaAjustesSlotterAjustePressaoDoEntalheDianteiro};
      \item[\ding{\dingNumber}] \ref{telaAjustesSlotterAjustePressaoVincos} - \nameref{telaAjustesSlotterAjustePressaoVincos};
      \item[\ding{\dingNumber}] \ref{telaAjustesPerfuradoraAjustaPressaoRoloPuxador} - \nameref{telaAjustesPerfuradoraAjustaPressaoRoloPuxador};
      \item[\ding{\dingNumber}] \ref{telaAjustesDobra2AjustaPressaoRoloEntrada} - \nameref{telaAjustesDobra2AjustaPressaoRoloEntrada};
      \item[\ding{\dingNumber}] \ref{telaAjustesDobra2AjustaPressaoRoloSaida} - \nameref{telaAjustesDobra2AjustaPressaoRoloSaida};
      \item[\ding{\dingNumber}] \ref{telaAjustesContagemAjustaPressaoRoloEntrada} - \nameref{telaAjustesContagemAjustaPressaoRoloEntrada}.
      
      \newpage
      \thispagestyle{fancy}
      \vspace*{50 pt}

      \else

      \item[\ding{\dingNumber}] \ref{sec:telaConfiguracoesAlimentacaoPressaoRoloPuxador} - \nameref{sec:telaConfiguracoesAlimentacaoPressaoRoloPuxador};

      \fi
    \end{pressureAdjustment}

    
  
  \ifmachineType
  \item[\ding{\dingNumber}] Ajustar a dobra e a contagem de acordo com as dimensões da caixa;Ver: 
  %cria uma nova lista com o nome ajustes de dobra e contagem
  \newlist {foldCount}{itemize}{1}
  %define as propriedades da lista
  \setlist [foldCount]{label={}, nosep, before=\vspace{10pt}, after=\vspace{10pt}}
  %inicia a lista de itens
  \begin{foldCount}
    

    \item[\ding{\dingNumber}] \ref{telaAjustesDobraAjustaPosicaoAxialAplicadorCola} - \nameref{telaAjustesDobraAjustaPosicaoAxialAplicadorCola};
    \item[\ding{\dingNumber}] \ref{telaAjustesDobraAjustaPosicaoAxialViga1} - \nameref{telaAjustesDobraAjustaPosicaoAxialViga1};
    \item[\ding{\dingNumber}] \ref{telaAjustesDobraAjustaPosicaoAxialViga2} - \nameref{telaAjustesDobraAjustaPosicaoAxialViga2};
    \item[\ding{\dingNumber}] \ref{telaAjustesDobraCalculaPosicaoRadialAplicacaoCola} - \nameref{telaAjustesDobraCalculaPosicaoRadialAplicacaoCola};
    \item[\ding{\dingNumber}] \ref{telaAjustesContagemAjustaSeparaPacoteAlturaPilha} - \nameref{telaAjustesContagemAjustaSeparaPacoteAlturaPilha};
    \item[\ding{\dingNumber}] \ref{telaAjustesContagemAjustaQuantidadeCaixasPilha} - \nameref{telaAjustesContagemAjustaQuantidadeCaixasPilha};
    \item[\ding{\dingNumber}] \ref{telaAjustesContagemAjustaSeparacaoPacoteOffsetPilha} - \nameref{telaAjustesContagemAjustaSeparacaoPacoteOffsetPilha};
    \item[\ding{\dingNumber}] \ref{telaAjustesContagemAjustaOffsetEspessuraCaixaDobrada} - \nameref{telaAjustesContagemAjustaOffsetEspessuraCaixaDobrada};
    \item[\ding{\dingNumber}] \ref{telaAjustesContagemFormacaoPilhaOffset} - \nameref{telaAjustesContagemFormacaoPilhaOffset}.
    
    
  \end{foldCount}

  \else

  \item[\ding{\dingNumber}] Ajustar o batedor e o empilhador de acordo com as dimensões da caixa;Ver:
  
  \newlist {stacker}{itemize}{1}
  \setlist [stacker]{label={}, nosep, before=\vspace{10pt}, after=\vspace{10pt}}

  \begin{stacker}

    \item[\ding{\dingNumber}] \ref{sec:telaCalculoAutomaticoGuiasDadosCaixa} - \nameref{sec:telaCalculoAutomaticoGuiasDadosCaixa};
    \item[\ding{\dingNumber}] \ref{sec:telaAjustesEmpilhadorAlturaAvancoElevadorEntrada} - \nameref{sec:telaAjustesEmpilhadorAlturaAvancoElevadorEntrada};
    \item[\ding{\dingNumber}] \ref{sec:telaAjustesEmpilhadorAlturaBatente} - \nameref{sec:telaAjustesEmpilhadorAlturaBatente}
    \item[\ding{\dingNumber}] \ref{sec:telaAjustesEmpilhadorEscovaSaidaEmpilhador} - \nameref{sec:telaAjustesEmpilhadorEscovaSaidaEmpilhador}
    \item[\ding{\dingNumber}] \ref{sec:telaAjustesEmpilhadorNoCrushEntradaTRP2} - \nameref{sec:telaAjustesEmpilhadorNoCrushEntradaTRP2}
    \item[\ding{\dingNumber}] \ref{sec:telaAjustesEmpilhadorQuantidadeCaixasPilha} - \nameref{sec:telaAjustesEmpilhadorQuantidadeCaixasPilha}
    \item[\ding{\dingNumber}] \ref{sec:telaAjustesEmpilhadorPosicaoParadaPilha} - \nameref{sec:telaAjustesEmpilhadorPosicaoParadaPilha}
    
  \end{stacker}
    
  \fi
  \item[\ding{\dingNumber}] Ajustar a pressão de corte conforme o ferramental e o desgaste da manta; Ver: 
  \ifmachineType
  \ref{telaAjustesPerfuradoraAjustePressaoPortaManta} - \nameref{telaAjustesPerfuradoraAjustePressaoPortaManta};
  \ref{telaAjustesPerfuradoraAlteraPerimetroManta} - \nameref{telaAjustesPerfuradoraAlteraPerimetroManta};
  \else
  \ref{sec:comandosCorteVincoAjustePressaoPortaManta} - \nameref{sec:comandosCorteVincoAjustePressaoPortaManta};
  \ref{sec:telaAjustesCorteEVincoPerimetroManta} - \nameref{sec:telaAjustesCorteEVincoPerimetroManta};
  \fi
  \item[\ding{\dingNumber}] Fazer ponto zero geral da máquina; Ver: 
  \ifmachineType
  \ref{telaComandosMaquinaPontoZeroGeral} - \nameref{telaComandosMaquinaPontoZeroGeral};
  \else
  \ref{sec:comandosMaquinaexecutaPontoZeroGeral} - \nameref{sec:comandosMaquinaexecutaPontoZeroGeral};
  \fi
  \item[\ding{\dingNumber}] Ligar a máquina e fazer o teste de impressão; Ver: 
  \ifmachineType
  \ref{telaComandoAlimentacaoConfiguracaoDeAlimentacaoDeCaixasModoManual} - \nameref{telaComandoAlimentacaoConfiguracaoDeAlimentacaoDeCaixasModoManual};
  \else
  \ref{sec:telaComandoAlimentacaoAlimentacaoManualDaMaquina} - \nameref{sec:telaComandoAlimentacaoAlimentacaoManualDaMaquina};
  \fi
  \item[\ding{\dingNumber}] Ajustar os registros de impressão; Ver: 
  \ifmachineType
  \ref{ajustaRegistroRoloPortaCliche} - \nameref{ajustaRegistroRoloPortaCliche};
  \else
  \ref{sec:telaAjustesImpressoraRadialPortaCliche} - \nameref{sec:telaAjustesImpressoraRadialPortaCliche};
  \fi
  \ifmachineType
  \item[\ding{\dingNumber}] Ajustar os registros dos entalhes; Ver: \ref{telaAjustesSlotterAjusteRegistroEntalheDianteiro} - \nameref{telaAjustesSlotterAjusteRegistroEntalheDianteiro} e \ref{telaAjustesSlotterAjusteRegistroEntalheTraseiro} - \nameref{telaAjustesSlotterAjusteRegistroEntalheTraseiro};
  \item[\ding{\dingNumber}] Ajustar os registros da perfuradora; Ver: \ref{telaAjustesPerfuradoraAjusteRegistroPortaFerramenta} - \nameref{telaAjustesPerfuradoraAjusteRegistroPortaFerramenta};
  \else
  \item[\ding{\dingNumber}] Ajustar o resgistro do corte; Ver: \ref{sec:telaAjustesCorteEVincoRadialPortaFerramenta} - \nameref{sec:telaAjustesCorteEVincoRadialPortaFerramenta};
  \fi
  \item[\ding{\dingNumber}] Ajustar o ajuste fino das pressões de acordo com a impressão;
  \item[\ding{\dingNumber}] Iniciar a produção.
  
\end{procedureAdjustmentNoRecipe}

\ifmachineType
\else
  \newpage
  \thispagestyle{fancy}
  \vspace*{40 pt}
\fi

\subsection{Procedimento para ajuste da máquina com receita}

%criar uma nova lista de itens com o nome procedureAdjustmentRecipe
\newlist {procedureAdjustmentRecipe}{itemize}{1}
%define as propriedades da lista
\setlist [procedureAdjustmentRecipe]{label={}, nosep, after=\vspace{10pt}}

%inicia a lista de itens
\begin{procedureAdjustmentRecipe}
  
  \item[\ding{\dingNumber}] Ajustar o carro esquadrejador e as guias de acordo com o tamanho da chapa; Ver: \ref{ihmAlimentacaoTelaAjustesPagina1} - \nameref{ihmAlimentacaoTelaAjustesPagina1};
  \item[\ding{\dingNumber}] Colocar clichê e tintas nas impressoras que serão usadas;
  \item[\ding{\dingNumber}] Colocar o ferramental na máquina;
  \item[\ding{\dingNumber}] Carregar a receita, ou criar uma nova receita; Ver: 
  \ifmachineType
  \ref{telaVisualizacaoPedidos} - \nameref{telaVisualizacaoPedidos};
  \else
  \ref{sec:telaVisualizacaoReceitas} - \nameref{sec:telaVisualizacaoReceitas};
  \fi
  \item[\ding{\dingNumber}] Ajustar a pressão de corte conforme o ferramental e o desgaste da manta; Ver: 
  \ifmachineType
  \ref{telaAjustesPerfuradoraAjustePressaoPortaManta} - \nameref{telaAjustesPerfuradoraAjustePressaoPortaManta};
  \ref{telaAjustesPerfuradoraAlteraPerimetroManta} - \nameref{telaAjustesPerfuradoraAlteraPerimetroManta};
  \else
  \ref{sec:comandosCorteVincoAjustePressaoPortaManta} - \nameref{sec:comandosCorteVincoAjustePressaoPortaManta};
  \ref{sec:telaAjustesCorteEVincoPerimetroManta} - \nameref{sec:telaAjustesCorteEVincoPerimetroManta};
  \fi
  \item[\ding{\dingNumber}] Ligar a máquina e fazer o teste de impressão;
  \item[\ding{\dingNumber}] Ajustar os registros de impressão; Ver: 
  \ifmachineType
  \ref{ajustaRegistroRoloPortaCliche} - \nameref{ajustaRegistroRoloPortaCliche};
  \else
  \ref{sec:segundaTelaGeralAjustesImpressorasRadialPortaCliche} - \nameref{sec:segundaTelaGeralAjustesImpressorasRadialPortaCliche};
  \fi
  \ifmachineType
  \item[\ding{\dingNumber}] Ajustar os registros dos entalhes; Ver: \ref{telaAjustesSlotterAjusteRegistroEntalheDianteiro} - \nameref{telaAjustesSlotterAjusteRegistroEntalheDianteiro} e \ref{telaAjustesSlotterAjusteRegistroEntalheTraseiro} - \nameref{telaAjustesSlotterAjusteRegistroEntalheTraseiro};
  \item[\ding{\dingNumber}] Ajustar os registros da perfuradora; Ver: \ref{telaAjustesPerfuradoraAjusteRegistroPortaFerramenta} - \nameref{telaAjustesPerfuradoraAjusteRegistroPortaFerramenta};
  \else
  \item[\ding{\dingNumber}] Ajustar o resgistro do corte; Ver: \ref{sec:telaAjustesCorteEVincoRadialPortaFerramenta} - \nameref{sec:telaAjustesCorteEVincoRadialPortaFerramenta};
  \fi
  \item[\ding{\dingNumber}] Ajustar as pressões de acordo com a impressão;
  \item[\ding{\dingNumber}] Salvar ajustes finos;
  \item[\ding{\dingNumber}] Iniciar a produção.
  
\end{procedureAdjustmentRecipe}

\ifmachineType
  \newpage
  \thispagestyle{fancy}
  \vspace*{40 pt}
\fi

\subsection{Procedimento para resolução de embuchamentos no setor 1}

%criar uma nova lista de itens com o nome procedureFixingIsuesSector1

\newlist {procedureFixingIsuesSector1}{itemize}{1}
%define as propriedades da lista
\setlist [procedureFixingIsuesSector1]{label={}, nosep, after=\vspace{10pt}}

%inicia a lista de itens
%procedimentos para desembuchamento do setor 1
\begin{procedureFixingIsuesSector1}

  \item[\ding{\dingNumber}] Desative o controle de embuchamento; Ver: 
  \ifmachineType
  \ref{telaComandosMaquinaLigaControleDeEmbuchamento} - \nameref{telaComandosMaquinaLigaControleDeEmbuchamento};
  \else
  \ref{sec:comandosMaquinaLigaControleDoEmbuchamento} - \nameref{sec:comandosMaquinaLigaControleDoEmbuchamento};
  \fi
  \ifmachineType
  \item[\ding{\dingNumber}] Abra a tampa da unidade em que o embuchamento ocorreu;
  \else
  \item[\ding{\dingNumber}] Destrave a unidade em que o embuchamento ocorreu;
  \item[\ding{\dingNumber}] Abra a máquina;
  \fi
  \item[\ding{\dingNumber}] Remova o embuchamento;
  \ifmachineType
  \item[\ding{\dingNumber}] Feche a tampa da unidade;
  \else
  \item[\ding{\dingNumber}] Feche a máquina;
  \item[\ding{\dingNumber}] Trave a unidade;
  \fi
  \item[\ding{\dingNumber}] Resete as falhas;
  \item[\ding{\dingNumber}] Ative o controle de embuchamento.
  
\end{procedureFixingIsuesSector1}

%criar uma nova lista de itens com o nome procedureFixingIsuesSector2

\newlist {procedureFixingIsuesSector2}{itemize}{1}
%define as propriedades da lista
\setlist [procedureFixingIsuesSector2]{label={}, nosep, after=\vspace{10pt}}

%inicia a lista de itens
%procedimentos para desembuchamento do setor 2
\ifmachineType

\subsection{Procedimento para resolução de embuchamentos no setor 2}

\begin{procedureFixingIsuesSector2}

  \item[\ding{\dingNumber}] Caso o embuchamento necessite acessar a área para a sua resolução:
  
  %criar uma nova lista de itens com o nome procedureFixingIsuesSector2.1
  \newlist {procedureFixingIsuesSector2.1}{itemize}{1}
  %define as propriedades da lista
  \setlist [procedureFixingIsuesSector2.1]{label={}, nosep}

  %inicia a lista de itens
  \begin{procedureFixingIsuesSector2.1}
  
    \item[\ding{\dingNumber}] Desbloqueie a porta da unidade por meio da comutadora bloqueio local;
    \item[\ding{\dingNumber}] Acesse a área onde o embuchamento ocorreu;
    \item[\ding{\dingNumber}] Remova o embuchamento;
    \item[\ding{\dingNumber}] Feche a porta da unidade;
    \item[\ding{\dingNumber}] Retorne a comutadora bloqueio local para a posição normal de operação.
    
  \end{procedureFixingIsuesSector2.1}

  \item[\ding{\dingNumber}] Caso o embuchamento não necessite acessar a área para a sua resolução:
  
  %criar uma nova lista de itens com o nome procedureFixingIsuesSector2.2
  \newlist {procedureFixingIsuesSector2.2}{itemize}{1}
  %define as propriedades da lista
  \setlist [procedureFixingIsuesSector2.2]{label={}, nosep}

  %inicia a lista de itens
  \begin{procedureFixingIsuesSector2.2}
  
    \item[\ding{\dingNumber}] Aperte o botão de JOG até que o embuchamento seja removido, Ver: \ref{telaComandoDobraJOG} - \nameref{telaComandoDobraJOG};
    
  \end{procedureFixingIsuesSector2.2}
  
  \item[\ding{\dingNumber}] Resete as falhas;
  \item[\ding{\dingNumber}] Ative o controle de embuchamento.
  
\end{procedureFixingIsuesSector2}

\fi

\subsection{Procedimento para resolução de embuchamentos no setor \ifmachineType 3 \else 2 \fi}

\newlist {procedureFixingIsuesSector3}{itemize}{1}
%define as propriedades da lista
\setlist [procedureFixingIsuesSector3]{label={}, nosep, after=\vspace{10pt}}

%inicia a lista de itens
%procedimentos para desembuchamento do setor 2
\begin{procedureFixingIsuesSector3}

  \item[\ding{\dingNumber}] Habilite o modo embuchamento; Ver:
  \ifmachineType
  \ref{telaComandoContagemHabilitaModoEmbuchamento} - \nameref{telaComandoContagemHabilitaModoEmbuchamento};
  \else
  \ref{sec:telaComandosEmpilhadorHabilitaModoEmbuchamento} - \nameref{sec:telaComandosEmpilhadorHabilitaModoEmbuchamento};
  \fi
  \item[\ding{\dingNumber}] Descarte a pilha; Ver:
  \ifmachineType
  \ref{telaComandoContagemDescartaPilha} - \nameref{telaComandoContagemDescartaPilha};
  \else
  \ref{sec:telaComandosEmpilhadorDescartaPilha} - \nameref{sec:telaComandosEmpilhadorDescartaPilha};
  \fi
  \item[\ding{\dingNumber}] Pressione o botão de JOG até que \ifmachineType a dobra \else o empilhador \fi não tenha mais nenhuma caixa; Ver:
  \ifmachineType
  \ref{telaComandoContagemMenuJOGEixos} - \nameref{telaComandoContagemMenuJOGEixos};
  \else
  \ref{sec:telaComandosEmpilhadorHabilitaFuncaoJog} - \nameref{sec:telaComandosEmpilhadorHabilitaFuncaoJog};
  \fi
  \item[\ding{\dingNumber}] Descarte a pilha;
  \item[\ding{\dingNumber}] Desabilite o modo embuchamento;
  \item[\ding{\dingNumber}] Devolva os eixos para a posição inicial; Ver:
  \ifmachineType
  \ref{telaComandoContagemMovimentoInicialDosEixos} - \nameref{telaComandoContagemMovimentoInicialDosEixos};
  \else
  \ref{sec:telaComandosEmpilhadorMovimentoInicialEixos} - \nameref{sec:telaComandosEmpilhadorMovimentoInicialEixos};
  \fi
  \item[\ding{\dingNumber}] Resete as falhas.
  
\end{procedureFixingIsuesSector3}

\newpage
\thispagestyle{fancy}
\vspace*{40 pt}
  
\section{\large\MakeUppercase{{Lógica de segurança da máquina}}}

\ifmachineType
 A máquina é composta por unidades Alimentação, Impressoras, Slotter, Perfuradora, Dobra e Contagem, e é dividida em três zonas de segurança.
 O Setor 1 compreende a área da Alimentação até a Perfuradora, o Setor 2 compreende a Dobra e o Setor 3 compreende a Contagem.
\else
A máquina é composta por unidades Alimentação, Impressoras, Transporte, Corte e Vinco, Batedor e Empilhador, e é dividida em duas zonas de segurança.
 O Setor 1 compreende a área da Alimentação até a Corte e Vinco e o Setor 2 compreende o restante da máquina.
\fi

Os dispositivos de segurança são:
\vspace*{10pt}
%criar uma nova lista de itens com o nome safetyDevices

\newlist {safetyDevices}{itemize}{1}
%define as propriedades da lista
\setlist [safetyDevices]{label={}, nosep, after=\vspace*{10pt}}

%inicia a lista de itens

\begin{safetyDevices}

  \item[\ding{\dingNumber}] Botões de Emergência: Estão situados \ifmachineType \else tanto \fi no lado de comando \ifmachineType \else quanto no interior das unidades\fi. Quando pressionados, bloqueiam todos os eixos da máquina;
  \item[\ding{\dingNumber}] Chaves comutadoras: Servem para bloquear o equipamento durante a manutenção e setup;
  \ifmachineType
  \item[\ding{\dingNumber}] Sensores de barreira: Detectam a entrada de pessoas entre as unidades;
  \else
  \item[\ding{\dingNumber}] Sensores de barreira: Detectam a entrada de pessoas entre as unidades quando o equipamento está aberto. Há um sensor no lado comando para cada 
  unidade do setor 1 e um único sensor abrangendo todas as unidades no lado acionamento;
  \fi
  \item[\ding{\dingNumber}] Botão reset acesso interno: Deve ser pressionado toda vez que um acesso interno for detectado, em sua respectiva unidade. Caso o acesso 
  interno seja do lado acionamento o botão reset de todas as unidades deve ser pressionado e também o botão reset do lado acionamento;
  \ifmachineType
  \item[\ding{\dingNumber}] Pedaleiras de três estágios: São utilizadas para movimentar o rolo porta ferramenta durante a troca do ferramental. O terceiro estágio é de
  emergência e é resetado pelo botão azul em cima da pedaleira;
  \else
  \item[\ding{\dingNumber}] Pedaleiras de três estágios: São utilizadas para movimentar os rolos porta ferramenta e porta clichê durante a troca do ferramental.
  O terceiro estágio é de emergência e é resetado pelo botão azul em cima da pedaleira;
  \fi
  \ifmachineType
  \item[\ding{\dingNumber}] Travas de segurança nas portas das unidades: Devem ser liberadas através do bloqueio local para abrir;
  \item[\ding{\dingNumber}] Sensores de abertura das portas: Caso alguma porta seja aberta, todos os eixos do setor entram no estado de emergência
  impossibilitando seu funcionamento; com exceção dos rolos porta ferramenta e porta clichê que são movimentados através de pedaleiras e botões em baixa velocidade
  e incrementos pequenos acada toque do botão para permitir a troca do ferramental;
  \else
  \item[\ding{\dingNumber}] Botão abrir/fechar + botão auxiliar (bimanual): Necessários para abrir e fechar o equipamento;
  \item[\ding{\dingNumber}] Travas de segurança nas portas do Empilhador: Devem ser liberadas através do bloqueio global para abrir;
  \item[\ding{\dingNumber}] Sensores de abertura de portas dos paineis elétricos: Detectam a abertura de portas dos paineis das unidades caso seja detectado a abertura do
   painel é suspensa a força elétrica na mesma;
  \item[\ding{\dingNumber}] Sensores de abertura das portas do Empilhador: Caso alguma porta do empilhador seja aberta, todos os eixos do setor entram no estado de emergência. 
  impossibilitando o funcionamento do mesmo;
  \fi
  \item[\ding{\dingNumber}] Sensores de acoplamento das unidades: Detectam se as unidades estão acopladas corretamente. Caso não estejam, o setor entra em estado de emergência;
  \item[\ding{\dingNumber}] Botões bimanuais na alimentação: Garante a segurança do operador durante a alimentação da máquina.

\end{safetyDevices}

\ifmachineType
\subsection{Segurança ao abrir e fechar as portas}

Para abrir as portas da maquina, é necessário primeiro realizar o bloqueio local da máquina por meio das chaves comutadoras, após isso, as travas das portas da unidade 
serão liberadas. Para devolver o equipamento para o estado de operação, é necessário primeiro fechar as portas, devolver achave comutadora para a posição de 
operação, após isso, as travas das portas serão bloqueadas por fim resetar o acesso interno e o equipamento estará pronto para operar.

\else

\subsection{Segurança ao abrir e fechar a máquina}

Para abrir o equipamento é necessário primeiro desligar o acionamento atraves do respectivo botão em qualquer unidade, após isso, é necessário destravar a unidade 
desejada para abrir, por fim é necessário pressionar o botão de abrir e o botão auxiliar (bimanual) na unidade alimentação. A máquina vai soar um alarme e em seguida
vai iniciar o movimento pelo tempo em que os botões forem pressionados. Para fechar o equipamento é necessário primeiro pressionar o botão de fechar no batedor e outra
pessoa deve pressionar o botão auxiliar (bimanual) e o botão de fecha na unidade alimentação. A máquina vai soar um alarme e em seguida vai iniciar o movimento pelo tempo 
em que os botões forem pressionados.

\subsection{Segurança ao abrir e fechar as portas do empilhador}

Para abrir as portas do Empilhador, é necessário primeiro realizar o bloqueio global da máquina por meio das chaves comutadoras, após isso, as travas das portas do
Empilhador serão liberadas. Para devolver o equipamento para o estado de operação, é necessário primeiro fechar as portas, devolver a chave comutadora para a posição de 
operação, após isso, as travas das portas do Empilhador serão bloqueadas por fim resetar o acesso interno e o equipamento estará pronto para operar.

\newpage
\thispagestyle{fancy}
\vspace*{40 pt}

\fi

\subsection{Segurança ao entrar no interior da máquina}

Para acessar o interior da máquina é necessário primeiro realizar o bloqueio \ifmachineType local \else global \fi da máquina por meio das chaves comutadoras e levar a chave consigo, após realizar o
seu trabalho, é necessário devolver a chave para a posição de operação e resetar o acesso interno das unidades acessadas pelo lado comando, caso o acesso interno tenha
sido realizado pelo lado acionamento, é necessário resetar o acesso interno de todas as unidades no setor lado comando e também resetar o acesso interno do lado acionamento.

\ifmachineType
  \newpage
  \thispagestyle{fancy}
  \vspace*{40 pt}
\fi

\subsection{Segurança ao trocar ferramental}

A troca do ferramental (forma e clichês) é feita com o auxílio de \ifmachineType pedaleiras de três estágios ou botões\else pedaleiras de três estágios\fi, sendo que o 
eixo vai se mover uma pequena distância em baixa velocidade a cada toque \ifmachineType no segundo estágio da pedaleira ou no botão \else no segundo estágio da pedaleira \fi. 
O terceiro estágio é de emergência e é resetado pelo botão azul em cima da pedaleira.


\subsection{Segurança ao alimentar a máquina}

O equipamento ao identificar um nível baixo na pilha de papel, vai emitir um alarme na IHM e parar ou impedir a alimentação. Você pode alimentar a máquina com chapas de papelão
e para retomar a produção é necessário apertar o bimanual de alimentação e habilitar novamente a alimentação. Caso queira introduzir a pilha até o final, é necessário
apertar o bimanual de alimentação até a pilha acabar.

