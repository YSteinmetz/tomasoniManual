\thispagestyle{fancy}

\vspace*{40 pt}

\section{\large{PROCEDIMENTO PARA AJUSTE DA MÁQUINA SEM RECEITA}}

Atenção, ao ajustar a máquina sem receita não será possível salvar as configurações, para salvar as configurações é necessário ter uma receita carregada.
Não recomendamos ajustar a máquina sem receita, pois não será possível salvar as configurações e os seus finos serão perdidos ao carregar uma nova receita.

%cria uma nova lista de itens com o nome procedureAdjustmentNoRecipe
\newlist {procedureAdjustmentNoRecipe}{itemize}{1}
%define as propriedades da lista
\setlist [procedureAdjustmentNoRecipe]{label={}, nosep, before=\vspace{10pt}, after=\vspace{10pt}}

%inicia a lista de itens
\begin{procedureAdjustmentNoRecipe}
  \item[\ding{\dingNumber}] Ajustar o carro esquadrejador e as guias de acordo com o tamanho da chapa; Ver: \ref{ihmAlimentacaoTelaAjustesPagina1} - \nameref{ihmAlimentacaoTelaAjustesPagina1};
  \item[\ding{\dingNumber}] Colocar clichê e tintas nas impressoras que serão usadas;
  \item[\ding{\dingNumber}] Colocar o ferramental na máquina;
  \item[\ding{\dingNumber}] Habilitar as impressoras que serão usadas; Ver: 
  \ifmachineType
  \ref{telaComandoImpressoraHabilitaImpressora} - \nameref{telaComandoImpressoraHabilitaImpressora};
  \else
  \ref{sec:comandosImpressoraHabilitaImpressora} - \nameref{sec:comandosImpressoraHabilitaImpressora};
  \fi
  \item[\ding{\dingNumber}] Fazer os ajustes de pressão no anilox conforme a expessura do clichê; Ver: 
  \ifmachineType
  \ref{ajustaPressaoRoloAnilox} - \nameref{ajustaPressaoRoloAnilox};
  \else
  \ref{sec:telaAjustesImpressoraPressaoRoloAnilox} - \nameref{sec:telaAjustesImpressoraPressaoRoloAnilox};
  \fi
  \item[\ding{\dingNumber}] Ajustar a pressão da caixa de vácuo considerando a espessura do clichê e a espessura da chapa; Ver: 
  \ifmachineType
  \ref{ajustaPressaoCaixaVacuo} - \nameref{ajustaPressaoCaixaVacuo};
  \else
  \ref{sec:telaAjustesImpressoraPressaoCaixaVaco} - \nameref{sec:telaAjustesImpressoraPressaoCaixaVaco};
  \fi
  \item[\ding{\dingNumber}] Realizar os outros ajustes de pressão de acordo com a espessura da chapa; Ver: 
    %Criar nova lista com nome pressureAdjustment
    \newlist {pressureAdjustment}{itemize}{1}
    %define as propriedades da lista
    \setlist [pressureAdjustment]{label={}, nosep, before=\vspace{10pt}, after=\vspace{10pt}}
    %inicia a lista de itens

    \begin{pressureAdjustment}
      \ifmachineType

      \item[\ding{\dingNumber}] \ref{telaAjustesAlimentacaoPressaoRoloPuxador} - \nameref{telaAjustesAlimentacaoPressaoRoloPuxador};
      \item[\ding{\dingNumber}] \ref{telaAjustesSlotterAjustePressaoDoAmassador} - \nameref{telaAjustesSlotterAjustePressaoDoAmassador};
      \item[\ding{\dingNumber}] \ref{telaAjustesSlotterAjustePressaoDoEntalheTraseiro} - \nameref{telaAjustesSlotterAjustePressaoDoEntalheTraseiro};
      \item[\ding{\dingNumber}] \ref{telaAjustesSlotterAjustePressaoDoEntalheDianteiro} - \nameref{telaAjustesSlotterAjustePressaoDoEntalheDianteiro};
      \item[\ding{\dingNumber}] \ref{telaAjustesSlotterAjustePressaoVincos} - \nameref{telaAjustesSlotterAjustePressaoVincos};
      \item[\ding{\dingNumber}] \ref{telaAjustesPerfuradoraAjustaPressaoRoloPuxador} - \nameref{telaAjustesPerfuradoraAjustaPressaoRoloPuxador};
      \item[\ding{\dingNumber}] \ref{telaAjustesDobra2AjustaPressaoRoloEntrada} - \nameref{telaAjustesDobra2AjustaPressaoRoloEntrada};
      \item[\ding{\dingNumber}] \ref{telaAjustesDobra2AjustaPressaoRoloSaida} - \nameref{telaAjustesDobra2AjustaPressaoRoloSaida};
      \item[\ding{\dingNumber}] \ref{telaAjustesContagemAjustaPressaoRoloEntrada} - \nameref{telaAjustesContagemAjustaPressaoRoloEntrada}.
      
      \newpage
      \thispagestyle{fancy}
      \vspace*{50 pt}

      \else

      \item[\ding{\dingNumber}] \ref{sec:telaConfiguracoesAlimentacaoPressaoRoloPuxador} - \nameref{sec:telaConfiguracoesAlimentacaoPressaoRoloPuxador};

      \fi
    \end{pressureAdjustment}

    
  
  \ifmachineType
  \item[\ding{\dingNumber}] Ajustar a dobra e a contagem de acordo com as dimensões da caixa;Ver: 
  %cria uma nova lista com o nome ajustes de dobra e contagem
  \newlist {foldCount}{itemize}{1}
  %define as propriedades da lista
  \setlist [foldCount]{label={}, nosep, before=\vspace{10pt}, after=\vspace{10pt}}
  %inicia a lista de itens
  \begin{foldCount}
    

    \item[\ding{\dingNumber}] \ref{telaAjustesDobraAjustaPosicaoAxialAplicadorCola} - \nameref{telaAjustesDobraAjustaPosicaoAxialAplicadorCola};
    \item[\ding{\dingNumber}] \ref{telaAjustesDobraAjustaPosicaoAxialViga1} - \nameref{telaAjustesDobraAjustaPosicaoAxialViga1};
    \item[\ding{\dingNumber}] \ref{telaAjustesDobraAjustaPosicaoAxialViga2} - \nameref{telaAjustesDobraAjustaPosicaoAxialViga2};
    \item[\ding{\dingNumber}] \ref{telaAjustesDobraCalculaPosicaoRadialAplicacaoCola} - \nameref{telaAjustesDobraCalculaPosicaoRadialAplicacaoCola};
    \item[\ding{\dingNumber}] \ref{telaAjustesContagemAjustaSeparaPacoteAlturaPilha} - \nameref{telaAjustesContagemAjustaSeparaPacoteAlturaPilha};
    \item[\ding{\dingNumber}] \ref{telaAjustesContagemAjustaQuantidadeCaixasPilha} - \nameref{telaAjustesContagemAjustaQuantidadeCaixasPilha};
    \item[\ding{\dingNumber}] \ref{telaAjustesContagemAjustaSeparacaoPacoteOffsetPilha} - \nameref{telaAjustesContagemAjustaSeparacaoPacoteOffsetPilha};
    \item[\ding{\dingNumber}] \ref{telaAjustesContagemAjustaOffsetEspessuraCaixaDobrada} - \nameref{telaAjustesContagemAjustaOffsetEspessuraCaixaDobrada};
    \item[\ding{\dingNumber}] \ref{telaAjustesContagemFormacaoPilhaOffset} - \nameref{telaAjustesContagemFormacaoPilhaOffset}.
    
    
  \end{foldCount}

  \else

  \item[\ding{\dingNumber}] Ajustar o batedor e o empilhador de acordo com as dimensões da caixa;Ver:
  
  \newlist {stacker}{itemize}{1}
  \setlist [stacker]{label={}, nosep, before=\vspace{10pt}, after=\vspace{10pt}}

  \begin{stacker}

    \item[\ding{\dingNumber}] \ref{sec:telaCalculoAutomaticoGuiasDadosCaixa} - \nameref{sec:telaCalculoAutomaticoGuiasDadosCaixa};
    \item[\ding{\dingNumber}] \ref{sec:telaAjustesEmpilhadorAlturaAvancoElevadorEntrada} - \nameref{sec:telaAjustesEmpilhadorAlturaAvancoElevadorEntrada};
    \item[\ding{\dingNumber}] \ref{sec:telaAjustesEmpilhadorAlturaBatente} - \nameref{sec:telaAjustesEmpilhadorAlturaBatente}
    \item[\ding{\dingNumber}] \ref{sec:telaAjustesEmpilhadorEscovaSaidaEmpilhador} - \nameref{sec:telaAjustesEmpilhadorEscovaSaidaEmpilhador}
    \item[\ding{\dingNumber}] \ref{sec:telaAjustesEmpilhadorNoCrushEntradaTRP2} - \nameref{sec:telaAjustesEmpilhadorNoCrushEntradaTRP2}
    \item[\ding{\dingNumber}] \ref{sec:telaAjustesEmpilhadorQuantidadeCaixasPilha} - \nameref{sec:telaAjustesEmpilhadorQuantidadeCaixasPilha}
    \item[\ding{\dingNumber}] \ref{sec:telaAjustesEmpilhadorPosicaoParadaPilha} - \nameref{sec:telaAjustesEmpilhadorPosicaoParadaPilha}
    
  \end{stacker}
    
  \fi
  \item[\ding{\dingNumber}] Ajustar a pressão de corte conforme o ferramental e o desgaste da manta; Ver: 
  \ifmachineType
  \ref{telaAjustesPerfuradoraAjustePressaoPortaManta} - \nameref{telaAjustesPerfuradoraAjustePressaoPortaManta};
  \ref{telaAjustesPerfuradoraAlteraPerimetroManta} - \nameref{telaAjustesPerfuradoraAlteraPerimetroManta};
  \else
  \ref{sec:comandosCorteVincoAjustePressaoPortaManta} - \nameref{sec:comandosCorteVincoAjustePressaoPortaManta};
  \ref{sec:telaAjustesCorteEVincoPerimetroManta} - \nameref{sec:telaAjustesCorteEVincoPerimetroManta};
  \fi
  \item[\ding{\dingNumber}] Fazer ponto zero geral da máquina; Ver: 
  \ifmachineType
  \ref{telaComandosMaquinaPontoZeroGeral} - \nameref{telaComandosMaquinaPontoZeroGeral};
  \else
  \ref{sec:comandosMaquinaexecutaPontoZeroGeral} - \nameref{sec:comandosMaquinaexecutaPontoZeroGeral};
  \fi
  \item[\ding{\dingNumber}] Ligar a máquina e fazer o teste de impressão; Ver: 
  \ifmachineType
  \ref{telaComandoAlimentacaoConfiguracaoDeAlimentacaoDeCaixasModoManual} - \nameref{telaComandoAlimentacaoConfiguracaoDeAlimentacaoDeCaixasModoManual};
  \else
  \ref{sec:telaComandoAlimentacaoAlimentacaoManualDaMaquina} - \nameref{sec:telaComandoAlimentacaoAlimentacaoManualDaMaquina};
  \fi
  \item[\ding{\dingNumber}] Ajustar os registros de impressão; Ver: 
  \ifmachineType
  \ref{ajustaRegistroRoloPortaCliche} - \nameref{ajustaRegistroRoloPortaCliche};
  \else
  \ref{sec:telaAjustesImpressoraRadialPortaCliche} - \nameref{sec:telaAjustesImpressoraRadialPortaCliche};
  \fi
  \ifmachineType
  \item[\ding{\dingNumber}] Ajustar os registros dos entalhes; Ver: \ref{telaAjustesSlotterAjusteRegistroEntalheDianteiro} - \nameref{telaAjustesSlotterAjusteRegistroEntalheDianteiro} e \ref{telaAjustesSlotterAjusteRegistroEntalheTraseiro} - \nameref{telaAjustesSlotterAjusteRegistroEntalheTraseiro};
  \item[\ding{\dingNumber}] Ajustar os registros da perfuradora; Ver: \ref{telaAjustesPerfuradoraAjusteRegistroPortaFerramenta} - \nameref{telaAjustesPerfuradoraAjusteRegistroPortaFerramenta};
  \else
  \item[\ding{\dingNumber}] Ajustar o resgistro do corte; Ver: \ref{sec:telaAjustesCorteEVincoRadialPortaFerramenta} - \nameref{sec:telaAjustesCorteEVincoRadialPortaFerramenta};
  \fi
  \item[\ding{\dingNumber}] Ajustar o ajuste fino das pressões de acordo com a impressão;
  \item[\ding{\dingNumber}] Iniciar a produção.
  
\end{procedureAdjustmentNoRecipe}

\ifmachineType
\else
  \newpage
  \thispagestyle{fancy}
  \vspace*{40 pt}
\fi

\section{\large{PROCEDIMENTO PARA AJUSTE DA MÁQUINA COM RECEITA}}

%criar uma nova lista de itens com o nome procedureAdjustmentRecipe
\newlist {procedureAdjustmentRecipe}{itemize}{1}
%define as propriedades da lista
\setlist [procedureAdjustmentRecipe]{label={}, nosep, after=\vspace{10pt}}

%inicia a lista de itens
\begin{procedureAdjustmentRecipe}
  
  \item[\ding{\dingNumber}] Ajustar o carro esquadrejador e as guias de acordo com o tamanho da chapa; Ver: \ref{ihmAlimentacaoTelaAjustesPagina1} - \nameref{ihmAlimentacaoTelaAjustesPagina1};
  \item[\ding{\dingNumber}] Colocar clichê e tintas nas impressoras que serão usadas;
  \item[\ding{\dingNumber}] Colocar o ferramental na máquina;
  \item[\ding{\dingNumber}] Carregar a receita, ou criar uma nova receita; Ver: 
  \ifmachineType
  \ref{telaVisualizacaoPedidos} - \nameref{telaVisualizacaoPedidos};
  \else
  \ref{sec:telaVisualizacaoReceitas} - \nameref{sec:telaVisualizacaoReceitas};
  \fi
  \item[\ding{\dingNumber}] Ajustar a pressão de corte conforme o ferramental e o desgaste da manta; Ver: 
  \ifmachineType
  \ref{telaAjustesPerfuradoraAjustePressaoPortaManta} - \nameref{telaAjustesPerfuradoraAjustePressaoPortaManta};
  \ref{telaAjustesPerfuradoraAlteraPerimetroManta} - \nameref{telaAjustesPerfuradoraAlteraPerimetroManta};
  \else
  \ref{sec:comandosCorteVincoAjustePressaoPortaManta} - \nameref{sec:comandosCorteVincoAjustePressaoPortaManta};
  \ref{sec:telaAjustesCorteEVincoPerimetroManta} - \nameref{sec:telaAjustesCorteEVincoPerimetroManta};
  \fi
  \item[\ding{\dingNumber}] Ligar a máquina e fazer o teste de impressão;
  \item[\ding{\dingNumber}] Ajustar os registros de impressão; Ver: 
  \ifmachineType
  \ref{ajustaRegistroRoloPortaCliche} - \nameref{ajustaRegistroRoloPortaCliche};
  \else
  \ref{sec:segundaTelaGeralAjustesImpressorasRadialPortaCliche} - \nameref{sec:segundaTelaGeralAjustesImpressorasRadialPortaCliche};
  \fi
  \ifmachineType
  \item[\ding{\dingNumber}] Ajustar os registros dos entalhes; Ver: \ref{telaAjustesSlotterAjusteRegistroEntalheDianteiro} - \nameref{telaAjustesSlotterAjusteRegistroEntalheDianteiro} e \ref{telaAjustesSlotterAjusteRegistroEntalheTraseiro} - \nameref{telaAjustesSlotterAjusteRegistroEntalheTraseiro};
  \item[\ding{\dingNumber}] Ajustar os registros da perfuradora; Ver: \ref{telaAjustesPerfuradoraAjusteRegistroPortaFerramenta} - \nameref{telaAjustesPerfuradoraAjusteRegistroPortaFerramenta};
  \else
  \item[\ding{\dingNumber}] Ajustar o resgistro do corte; Ver: \ref{sec:telaAjustesCorteEVincoRadialPortaFerramenta} - \nameref{sec:telaAjustesCorteEVincoRadialPortaFerramenta};
  \fi
  \item[\ding{\dingNumber}] Ajustar as pressões de acordo com a impressão;
  \item[\ding{\dingNumber}] Salvar ajustes finos;
  \item[\ding{\dingNumber}] Iniciar a produção.
  
\end{procedureAdjustmentRecipe}

\ifmachineType
  \newpage
  \thispagestyle{fancy}
  \vspace*{40 pt}
\fi

\section{\large{PROCEDIMENTO PARA RESOLUÇÃO DE EMBUCHAMENTOS NO SETOR 1}}

%criar uma nova lista de itens com o nome procedureFixingIsuesSector1

\newlist {procedureFixingIsuesSector1}{itemize}{1}
%define as propriedades da lista
\setlist [procedureFixingIsuesSector1]{label={}, nosep, after=\vspace{10pt}}

%inicia a lista de itens
%procedimentos para desembuchamento do setor 1
\begin{procedureFixingIsuesSector1}

  \item[\ding{\dingNumber}] Desative o controle de embuchamento; Ver: 
  \ifmachineType
  \ref{telaComandosMaquinaLigaControleDeEmbuchamento} - \nameref{telaComandosMaquinaLigaControleDeEmbuchamento};
  \else
  \ref{sec:comandosMaquinaLigaControleDoEmbuchamento} - \nameref{sec:comandosMaquinaLigaControleDoEmbuchamento};
  \fi
  \ifmachineType
  \item[\ding{\dingNumber}] Destrave a unidade em que o embuchamento ocorreu;
  \item[\ding{\dingNumber}] Abra a máquina;
  \fi
  \item[\ding{\dingNumber}] Remova o embuchamento;
  \ifmachineType
  \item[\ding{\dingNumber}] Feche a máquina;
  \item[\ding{\dingNumber}] Trave a unidade;
  \fi
  \item[\ding{\dingNumber}] Resete as falhas;
  \item[\ding{\dingNumber}] Ative o controle de embuchamento.
  
\end{procedureFixingIsuesSector1}

\section{\large{PROCEDIMENTO PARA RESOLUÇÃO DE EMBUCHAMENTOS NO SETOR 2}}

%criar uma nova lista de itens com o nome procedureFixingIsuesSector2

\newlist {procedureFixingIsuesSector2}{itemize}{1}
%define as propriedades da lista
\setlist [procedureFixingIsuesSector2]{label={}, nosep, after=\vspace{10pt}}

%inicia a lista de itens
%procedimentos para desembuchamento do setor 2
\begin{procedureFixingIsuesSector2}

  \item[\ding{\dingNumber}] Habilite o modo embuchamento; Ver:
  \ifmachineType
  \ref{telaComandoContagemHabilitaModoEmbuchamento} - \nameref{telaComandoContagemHabilitaModoEmbuchamento};
  \else
  \ref{sec:telaComandosEmpilhadorHabilitaModoEmbuchamento} - \nameref{sec:telaComandosEmpilhadorHabilitaModoEmbuchamento};
  \fi
  \item[\ding{\dingNumber}] Descarte a pilha; Ver:
  \ifmachineType
  \ref{telaComandoContagemDescartaPilha} - \nameref{telaComandoContagemDescartaPilha};
  \else
  \ref{sec:telaComandosEmpilhadorDescartaPilha} - \nameref{sec:telaComandosEmpilhadorDescartaPilha};
  \fi
  \item[\ding{\dingNumber}] Pressione o botão de JOG até que a dobra não tenha mais nenhuma caixa; Ver:
  \ifmachineType
  \ref{telaComandoContagemMenuJOGEixos} - \nameref{telaComandoContagemMenuJOGEixos};
  \else
  \ref{sec:telaComandosEmpilhadorHabilitaFuncaoJog} - \nameref{sec:telaComandosEmpilhadorHabilitaFuncaoJog};
  \fi
  \item[\ding{\dingNumber}] Descarte a pilha;
  \item[\ding{\dingNumber}] Desabilite o modo embuchamento;
  \item[\ding{\dingNumber}] Devolva os eixos para a posição inicial; Ver:
  \ifmachineType
  \ref{telaComandoContagemMovimentoInicialDosEixos} - \nameref{telaComandoContagemMovimentoInicialDosEixos};
  \else
  \ref{sec:telaComandosEmpilhadorMovimentoInicialEixos} - \nameref{sec:telaComandosEmpilhadorMovimentoInicialEixos};
  \fi
  \item[\ding{\dingNumber}] Resete as falhas.
  
\end{procedureFixingIsuesSector2}
  



