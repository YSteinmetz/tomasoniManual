\thispagestyle{fancy}

\vspace*{40 pt}

\section{\large{PROCEDIMENTO PARA AJUSTE DA MÁQUINA SEM RECEITA}}

Atenção, ao ajustar a máquina sem receita não será possível salvar as configurações, para salvar as configurações é necessário ter uma receita carregada.
Não recomendamos ajustar a máquina sem receita, pois não será possível salvar as configurações e os seus finos serão perdidos ao carregar uma nova receita.

%cria uma nova lista de itens com o nome procedureAdjustmentNoRecipe
\newlist {procedureAdjustmentNoRecipe}{itemize}{1}
%define as propriedades da lista
\setlist [procedureAdjustmentNoRecipe]{label={}, nosep, before=\vspace{10pt}, after=\vspace{10pt}}

%inicia a lista de itens
\begin{procedureAdjustmentNoRecipe}
  \item[\ding{\dingNumber}] Ajustar o carro esquadrejador e as guias de acordo com o tamanho da chapa; Ver: \ref{ihmAlimentacaoTelaAjustesPagina1} - \nameref{ihmAlimentacaoTelaAjustesPagina1};
  \item[\ding{\dingNumber}] Colocar clichê e tintas nas impressoras que serão usadas;
  \item[\ding{\dingNumber}] Colocar o ferramental na máquina;
  \item[\ding{\dingNumber}] Habilitar as impressoras que serão usadas; Ver: \ref{telaComandoImpressoraExecutaLavagemDeTinta} - \nameref{telaComandoImpressoraExecutaLavagemDeTinta};
  \item[\ding{\dingNumber}] Fazer os ajustes de pressão no anilox conforme a expessura do clichê; Ver: \ref{ajustaPressaoRoloAnilox} - \nameref{ajustaPressaoRoloAnilox};
  \item[\ding{\dingNumber}] Ajustar a pressão da caixa de vácuo considerando a espessura do clichê e a espessura da chapa; Ver: \ref{ajustaPressaoCaixaVacuo} - \nameref{ajustaPressaoCaixaVacuo};
  \item[\ding{\dingNumber}] Realizar os outros ajustes de pressão de acordo com a espessura da chapa; Ver: 
    %Criar nova lista com nome pressureAdjustment
    \newlist {pressureAdjustment}{itemize}{1}
    %define as propriedades da lista
    \setlist [pressureAdjustment]{label={}, nosep, before=\vspace{10pt}, after=\vspace{10pt}}
    %inicia a lista de itens

    \begin{pressureAdjustment}
      \item[\ding{\dingNumber}] \ref{telaAjustesAlimentacaoPressaoRoloPuxador} - \nameref{telaAjustesAlimentacaoPressaoRoloPuxador};
      \item[\ding{\dingNumber}] \ref{telaAjustesSlotterAjustePressaoDoAmassador} - \nameref{telaAjustesSlotterAjustePressaoDoAmassador};
      \item[\ding{\dingNumber}] \ref{telaAjustesSlotterAjustePressaoDoEntalheTraseiro} - \nameref{telaAjustesSlotterAjustePressaoDoEntalheTraseiro};
      \item[\ding{\dingNumber}] \ref{telaAjustesSlotterAjustePressaoDoEntalheDianteiro} - \nameref{telaAjustesSlotterAjustePressaoDoEntalheDianteiro};
      \item[\ding{\dingNumber}] \ref{telaAjustesSlotterAjustePressaoVincos} - \nameref{telaAjustesSlotterAjustePressaoVincos};
      \item[\ding{\dingNumber}] \ref{telaAjustesPerfuradoraAjustaPressaoRoloPuxador} - \nameref{telaAjustesPerfuradoraAjustaPressaoRoloPuxador};
      \item[\ding{\dingNumber}] \ref{telaAjustesDobra2AjustaPressaoRoloEntrada} - \nameref{telaAjustesDobra2AjustaPressaoRoloEntrada};
      \item[\ding{\dingNumber}] \ref{telaAjustesDobra2AjustaPressaoRoloSaida} - \nameref{telaAjustesDobra2AjustaPressaoRoloSaida};
      \item[\ding{\dingNumber}] \ref{telaAjustesContagemAjustaPressaoRoloEntrada} - \nameref{telaAjustesContagemAjustaPressaoRoloEntrada}.
    \end{pressureAdjustment}

    \newpage
    \thispagestyle{fancy}
    \vspace*{50 pt}

  \item[\ding{\dingNumber}] Ajustar a dobra e a contagem de acordo com as dimensões da caixa;Ver: 
  %cria uma nova lista com o nome ajustes de dobra e contagem
  \newlist {foldCount}{itemize}{1}
  %define as propriedades da lista
  \setlist [foldCount]{label={}, nosep, before=\vspace{10pt}, after=\vspace{10pt}}
  %inicia a lista de itens
  \begin{foldCount}
    \item[\ding{\dingNumber}] \ref{telaAjustesDobraAjustaPosicaoAxialAplicadorCola} - \nameref{telaAjustesDobraAjustaPosicaoAxialAplicadorCola};
    \item[\ding{\dingNumber}] \ref{telaAjustesDobraAjustaPosicaoAxialViga1} - \nameref{telaAjustesDobraAjustaPosicaoAxialViga1};
    \item[\ding{\dingNumber}] \ref{telaAjustesDobraAjustaPosicaoAxialViga2} - \nameref{telaAjustesDobraAjustaPosicaoAxialViga2};
    \item[\ding{\dingNumber}] \ref{telaAjustesDobraCalculaPosicaoRadialAplicacaoCola} - \nameref{telaAjustesDobraCalculaPosicaoRadialAplicacaoCola};
    \item[\ding{\dingNumber}] \ref{telaAjustesContagemAjustaSeparaPacoteAlturaPilha} - \nameref{telaAjustesContagemAjustaSeparaPacoteAlturaPilha};
    \item[\ding{\dingNumber}] \ref{telaAjustesContagemAjustaQuantidadeCaixasPilha} - \nameref{telaAjustesContagemAjustaQuantidadeCaixasPilha};
    \item[\ding{\dingNumber}] \ref{telaAjustesContagemAjustaSeparacaoPacoteOffsetPilha} - \nameref{telaAjustesContagemAjustaSeparacaoPacoteOffsetPilha};
    \item[\ding{\dingNumber}] \ref{telaAjustesContagemAjustaOffsetEspessuraCaixaDobrada} - \nameref{telaAjustesContagemAjustaOffsetEspessuraCaixaDobrada};
    \item[\ding{\dingNumber}] \ref{telaAjustesContagemFormacaoPilhaOffset} - \nameref{telaAjustesContagemFormacaoPilhaOffset}.
  \end{foldCount}
  \item[\ding{\dingNumber}] Ajustar a pressão de corte conforme o ferramental e o desgaste da manta; Ver: \ref{telaAjustesPerfuradoraAjustePressaoPortaManta} - \nameref{telaAjustesPerfuradoraAjustePressaoPortaManta};
  \item[\ding{\dingNumber}] Fazer ponto zero geral da máquina; Ver: \ref{telaComandosMaquinaPontoZeroGeral} - \nameref{telaComandosMaquinaPontoZeroGeral};
  \item[\ding{\dingNumber}] Ligar a máquina e fazer o teste de impressão; Ver: \ref{telaComandoAlimentacaoConfiguracaoDeAlimentacaoDeCaixasModoManual} - \nameref{telaComandoAlimentacaoConfiguracaoDeAlimentacaoDeCaixasModoManual};
  \item[\ding{\dingNumber}] Ajustar os registros de impressão; Ver: \ref{ajustaRegistroRoloPortaCliche} - \nameref{ajustaRegistroRoloPortaCliche};
  \item[\ding{\dingNumber}] Ajustar os registros dos entalhes; Ver: \ref{telaAjustesSlotterAjusteRegistroEntalheDianteiro} - \nameref{telaAjustesSlotterAjusteRegistroEntalheDianteiro} e \ref{telaAjustesSlotterAjusteRegistroEntalheTraseiro} - \nameref{telaAjustesSlotterAjusteRegistroEntalheTraseiro};
  \item[\ding{\dingNumber}] Ajustar os registros da perfuradora; Ver: \ref{telaAjustesPerfuradoraAjusteRegistroPortaFerramenta} - \nameref{telaAjustesPerfuradoraAjusteRegistroPortaFerramenta};
  \item[\ding{\dingNumber}] Ajustar o ajuste fino das pressões de acordo com a impressão;
  \item[\ding{\dingNumber}] Iniciar a produção.
  
\end{procedureAdjustmentNoRecipe}

\section{\large{PROCEDIMENTO PARA AJUSTE DA MÁQUINA COM RECEITA}}

%criar uma nova lista de itens com o nome procedureAdjustmentRecipe
\newlist {procedureAdjustmentRecipe}{itemize}{1}
%define as propriedades da lista
\setlist [procedureAdjustmentRecipe]{label={}, nosep, before=\vspace{10pt}, after=\vspace{10pt}}

%inicia a lista de itens
\begin{procedureAdjustmentRecipe}
  
  \item[\ding{\dingNumber}] Ajustar o carro esquadrejador e as guias de acordo com o tamanho da chapa; Ver: \ref{ihmAlimentacaoTelaAjustesPagina1} - \nameref{ihmAlimentacaoTelaAjustesPagina1};
  \item[\ding{\dingNumber}] Colocar clichê e tintas nas impressoras que serão usadas;
  \item[\ding{\dingNumber}] Colocar o ferramental na máquina;
  \item[\ding{\dingNumber}] Carregar a receita, ou criar uma nova receita; Ver: \ref{telaVisualizacaoPedidos} - \nameref{telaVisualizacaoPedidos};
  \item[\ding{\dingNumber}] Ajustar a pressão de corte conforme o ferramental e o desgaste da manta; Ver: \ref{telaAjustesPerfuradoraAjustePressaoPortaManta} - \nameref{telaAjustesPerfuradoraAjustePressaoPortaManta};
  \item[\ding{\dingNumber}] Ligar a máquina e fazer o teste de impressão;
  \item[\ding{\dingNumber}] Ajustar os registros de impressão; Ver: \ref{ajustaRegistroRoloPortaCliche} - \nameref{ajustaRegistroRoloPortaCliche};
  \item[\ding{\dingNumber}] Ajustar os registros da perfuradora; Ver: \ref{telaAjustesPerfuradoraAjusteRegistroPortaFerramenta} - \nameref{telaAjustesPerfuradoraAjusteRegistroPortaFerramenta};
  \item[\ding{\dingNumber}] Ajustar as pressões de acordo com a impressão;
  \item[\ding{\dingNumber}] Salvar ajustes finos;
  \item[\ding{\dingNumber}] Iniciar a produção.
  
\end{procedureAdjustmentRecipe}

