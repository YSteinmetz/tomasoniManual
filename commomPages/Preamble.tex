\thispagestyle{fancy}

\vspace*{40 pt}

\section{\large{NORMAS APLICADAS NA MANUFATURA DA MÁQUINA}}
\label{sec:normas}

% Define a new list environment
\newlist{standartsMet}{itemize}{1}
\setlist[standartsMet]{label ={}, nosep, before=\vspace{10pt}, after=\vspace{10pt}}

\begin{standartsMet}
    \item[\ding{\dingNumber}] NR-10 - Segurança em Instalações e Serviços em Eletricidade;
    \item[\ding{\dingNumber}] NR-12 - Segurança no Trabalho em Máquinas e Equipamentos;
    \item[\ding{\dingNumber}] ABNT/CB-03 - Eletricidade;
    \item[\ding{\dingNumber}] ABNT/CB-04 - Máquinas e Equipamentos Mecânicos;
    \item[\ding{\dingNumber}] NBR-5410 - Instalações Elétricas de Baixa Tensão.
\end{standartsMet}

Nossos equipamentos atendem as normas de BPF (boas práticas de fabricação), reforçando a nossa qualidade no segmento.
Trabalhamos com ética, responsabilidade, comprometimento e com muito respeito para com os nossos clientes.

\section{\large{\MakeUppercase{Descrição Detalhada da Máquina e ou Equipamento}}}
\label{sec:machineDescription}

% Define a new list environment
\newlist{machinePartsList}{itemize}{1}
\setlist[machinePartsList]{label ={},nosep, before=\vspace{10pt}, after=\vspace{10pt}}

\begin{machinePartsList}
    \item[\ding{\dingNumber}] 1 - Unidade Alimentação;
    \item[\ding{\dingNumber}] \numberOfPrinters \space - Unidades Impressoras;
    \ifmachineType
        \item[\ding{\dingNumber}] 1 - Unidade Slotter;
        \ifunidadePerfuradora
        \item[\ding{\dingNumber}] 1 - Unidade Perfuradora;
        \fi
        \item[\ding{\dingNumber}] 1 - Unidade Dobra;
        \item[\ding{\dingNumber}] 1 - Unidade Contagem.
    \else
        \item[\ding{\dingNumber}] 1 - Unidade Transporte;
        \item[\ding{\dingNumber}] 1 - Unidade Corte e Vinco;
        \item[\ding{\dingNumber}] 1 - Unidade Batedor;
        \item[\ding{\dingNumber}] 1 - Unidade Empilhador.
    \fi
\end{machinePartsList}

Projetos Elétricos: cada unidade possui projeto elétrico representando os esquema de potência, controle e esquemático das
 funções de segurança. Os projetos são entregues juntamente com os manuais de operação, manutenção e
 peças de reposição.

 \ifmachineType

  \ifunidadePerfuradora
       Definição da utilização prevista para a Máquina:
 Máquina destinada à Produção de Caixas de Papelão tipo Maleta com possibilidade de furos e ou recortes.
 \else
 Definição da utilização prevista para a Máquina:
 Máquina destinada à Produção de Caixas de Papelão tipo Maleta.
    \fi
\else
???????????????????????????????????????????
\fi

\newpage

\thispagestyle{fancy}

\vspace*{40 pt}

\section{\large{\MakeUppercase{Descrição detalhada da Exposição à Riscos e Medidas de Segurança}}}

A Máquina é composta por por uma série de seguranças às peças móveis, desde sensores, proteções, etc, que nunca devem
ser bloqueados ou retirados da máquina, total ou parcialmente. As fontes de energia (energia elétrica, ar comprimido, etc)
somente deverão ser acessadas por pessoal de manutenção, com devido bloqueio e sinalização.
\textbf{VEJA OS PROCEDIMENTOS ESPECÍFICOS DA SUA EMPRESA.}

Os riscos apresentados durante a operação e ou manutenção do equipamento são:

\newlist{typeOfRisks}{itemize}{1}
\setlist[typeOfRisks]{label ={},nosep, before=\vspace{10pt}, after=\vspace{10pt}}


\begin{typeOfRisks}
\item[\ding{\dingNumber}] Riscos de Acidentes: Deslocamento das unidades, substituição de clichês, abertura e fechamento do tanque de cola pressurizado (quando houver), ajustes em geral da máquina;
\item[\ding{\dingNumber}] Riscos Ergonômicos: LER ao Alimentar máquina com chapas de papelão, reposição de tintas e cola;
\item[\ding{\dingNumber}] Riscos Físicos: Ruído;
\item[\ding{\dingNumber}] Riscos Químicos: Tinta, cola e poeira do papelão (devido aos cortes efetuados no papelão durante produção). Graxa e
óleos específicos para lubrificação da máquina (pessoal de manutenção);
\item[\ding{\dingNumber}] Riscos Biológicos: Não evidenciados;
\end{typeOfRisks}
Para evitar o risco a choques elétricos e a movimentos mecânicos dos acionamentos, todos os painéis e carenagens da
máquina possuem chaves, que são entregues aos pessoal de manutenção no momento do treinamento;
Todo o pessoal envolvido em operação e manutenção da Máquina deve estar devidamente capacitado técnicamente e ter
recebido treinamento específico, comprovado via certificado, fornecido pela Tomasoni no momento do treinamento e
entrega da máquina;
Pessoas alheias a operação e a manutenção que nunca receberam treinamento específicos devem ser impedidas de
realizar estas tarefas, por desconhecerem o funcionamento e os riscos de cada função;
\textbf{Para a diminuição à níveis aceitáveis e ou total exclusão dos riscos acima citados, são recomendadas a utilização de EPI's
em todas as etapas de operação e manutenção, e seguir as orientações contidos no PPRA E PCSMO da Empresa à qual a
máquina foi instalada.}

Para operação e ou manutenção, abertura e fechamento para abastecimento do tanque de cola (quando houver) temos
uma etiqueta com os procedimentos para realizar esta tarefa de forma segura. Esta etiqueta está instalada próxima do
tanque de cola;
Os usuários da máquina (operação e manutenção) nunca devem se colocar em posição de risco, subindo na máquina ou
entrando e permanecendo em partes com movimento;
Os usuários da máquina (operação e manutenção) nunca devem obstruir as saídas de ar dos ventiladores de vácuo,
tampouco excluir proteções que podem dar livre acesso a partes que contenham movimentos durante o seu processo
produtivo;
Os usuários da máquina (manutenção) munca devem iniciar seus trabalhos antes de bloquear e identificar o bloqueio de
todas as fontes de energia, a fim de impedir acionamentos involuntários;
\textbf{A Tomasoni indústria de Máquinas Ltda indica que a periodicidade para verificações de funcionamento adequado,
inspeções e manutenções dos sistemas de segurança da máquina não deve ultrapassar a marca de 30 dias;}
Em situação de Emergência, primeiramente deve ocorrer a parada total do equipamento. Após devem ser extinguidas ou
bloqueadas e identificadas as fontes de energia. Verificar o envolvimento humano ou não na situação de emergência,
tomar as providências. Verificar o defeito ou problema ocorrido na máquina. Resolver todas as avarias por menor que
sejam, tornando a máquina segura para reinício de operação e produção.
As Máquinas Tomasoni possuem vida útil estimada em 15 anos, desde que sejam verificadas as funcionalidades de
Operação, Produção, Manutenção e principalmente de Segurança aos indivíduos inseridos neste contexto.
\textbf{IMPORTANTE: As normas de segurança (NR12, NR10, etc) foram criadas com o objetivo de garantir a segurança aos
envolvidos no processo de Operação e Manutenção, bem como aumentar a produtividade.
}


\newpage
\thispagestyle{fancy}
\vspace*{50 pt}

As Máquinas Tomasoni foram projetadas para atender os mais rigorosos padrões de segurança para propiciar a maior segurança possível 
aos envolvidos nos processos de Projeto, Fabricação, Operação e Manutenção, DESDE QUE SEGUIDAS TODAS AS NORMAS E ORIENTAÇÕES CONTIDAS NESTE MANUAL.
Somente estão aptas a operar e ou auxiliar na operação e ou realizar
manutenções na máquina os colaboradores devidamente treinados e certificados.
Todos não inclusos nesta estão sujeitos a riscos desconhecidos, que podem
levar ao indivíduo sofrer acidentes.
Alterações modificando originalidade da máquina em qualquer aspecto, funcional
ou de segurança, leva a perda de garantia do equipamento.
A não observância de qualquer norma e ou orientação de segurança podem causar
acidentes, nas diversas escalas de gravidade. Os envolvidos em casos de
acidentes poderão vir a responder em juízo, civil e criminalmente conforme
previsto em lei, podendo acarretar em punições e perda de direitos.

\subsection{Regras gerais de segurança}
As Impressoras Tomasoni foram projetadas para operações seguras. O ajuste da máquina é motorizado e
totalmente controlado através do painel ihm (interface homem-máquina) e botões instalados no lado operacional da máquina.

A fim de evitar acidentes, todos que trabalham na máquina devem estar capacitados. Além do conhecimento, há
algumas regras gerais de segurança que devem ser seguidas. Esta parte do manual conterá estas regras:

\newlist{generalSafetyRules}{itemize}{1}
\setlist[generalSafetyRules]{label ={},nosep, before=\vspace{10pt}, after=\vspace{10pt}}

\begin{generalSafetyRules}
    \item[\ding{\dingNumber}] Utilizar sempre EPI's recomendados no PPRA e PCMSO da empresa para qualquer tipo de operação ou
manutenção na máquina;
    \item[\ding{\dingNumber}]  Complementando a utilização dos EPI's fornecidos pela empresa, os funcionários não devem utilizar
qualquer tipo de adorno (relógio, anel, pulseiras, correntes, brincos, etc) e também se tiver cabelos compridos
amarrá-los de maneira adequada para não causar riscos. Utilizar uniforme adequado, tomar cuidado com jalecos
de manga longa, etc;
    \item[\ding{\dingNumber}] Manutenções elétricas e ou mecânicas somente podem ser realizadas por pessoal capacitado
tecnicamente, autorizado pela empresa e que tenha recebido treinamento específico, segundo NR12;
    \item[\ding{\dingNumber}] Tenha todas as avarias da máquina sempre resolvidas, não importando o nível da avaria ou problema que a
máquina esteja apresentando;
    \item[\ding{\dingNumber}] Nunca abra as portas dos painéis elétricos. Somente pessoal habilitado pode ter acesso a esta área;
    \item[\ding{\dingNumber}] Nunca tente remover ou bloquear ou jumpear qualquer dispositivo de segurança na máquina;
    \item[\ding{\dingNumber}] Nunca tente limpar um embuchamento de caixas enquanto a máquina estiver em funcionamento;
    \item[\ding{\dingNumber}] A máquina não deve receber intervenções de manutenção durante o seu funcionamento. O recomendado é
desenergizar os painéis e utilizar dispositivos de segurança (cadeado e etiqueta identificadora de intervenção) na
chave geral, impedido o religamento inadequado;
    \item[\ding{\dingNumber}] Nunca tente remover proteções ou carenagens com a máquina em funcionamento;
    \item[\ding{\dingNumber}] Nunca tente subir sobre qualquer parte da máquina com a máquina em funcionamento;
    \item[\ding{\dingNumber}] Nunca tente obstruir a saída de ar dos ventiladores/exaustores com peças, mãos, braços ou qualquer
membro com a máquina em funcionamento;
    \item[\ding{\dingNumber}] Nunca utilize ar comprimido da máquina para limpeza corporal ou do uniforme;
    \item[\ding{\dingNumber}] Nunca toque em partes móveis com a máquina em funcionamento (lançador de caixas, rolos, esteiras,
elevador, lança do separador de pacotes, etc);
    \item[\ding{\dingNumber}] Nunca entre em partes móveis com a máquina em funcionamento (sobre ou embaixo do elevador, esteiras
de refile, lança do separador de pacotes, etc);
    \item[\ding{\dingNumber}] Respeitar as placas de sinalização e advertência instaladas em diversas partes da máquina e dos painéis
elétricos.



\newpage
\thispagestyle{fancy}
\vspace*{40 pt}


\end{generalSafetyRules}
\subsection{Partida da Máquina}

A partida da máquina deve ser feita com muito cuidado. Relacionamos abaixo alguns exemplos a seguir:

\newlist{generalSafetyToStartTheMachine}{itemize}{1}
\setlist[generalSafetyToStartTheMachine]{label ={},nosep, before=\vspace{10pt}, after=\vspace{10pt}}
\begin{generalSafetyToStartTheMachine}


    \item[\ding{\dingNumber}] Assegure-se que niguém esteja trabalhando na máquina antes de dar a partida;
    \item[\ding{\dingNumber}] Não permita que pessoas alheias (que não receberam treinamento e certificado) a operação fique em
contato com a máquina durante a partida e o funcionamento;
    \item[\ding{\dingNumber}] Verifique todas as tampas e proteções estejam fechadas adequadamente e que os painéis elétricos
estejam fechados;
    \item[\ding{\dingNumber}] Os membros da equipe devem estar em posição de maneira a supervisionar toda a máquina antes da
partida;
    \item[\ding{\dingNumber}] Seguindo as regras acima podemos dar a partida na máquina com segurança.

\end{generalSafetyToStartTheMachine}

\subsection{Deslocamentos das unidades}

Ao deslocar as unidades impressoras, perfuradora e unidade de contagem, é importante seguir alguns cuidados de segurança para evitar danos à máquina e ferimentos aos operadores:


\newlist{generalSafetyToMoveUnities}{itemize}{1}
\setlist[generalSafetyToMoveUnities]{label ={},nosep, before=\vspace{10pt}, after=\vspace{10pt}}
\begin{generalSafetyToMoveUnities}
    \item[\ding{\dingNumber}] A máquina possui alarme audiovisual para deslocamento das unidades (abertura e fechamento da máquina);
    \item[\ding{\dingNumber}] Para habilitar o funcionamento do sistema de deslocamento das unidades é necessário que a máquina esteja
parada;
    \item[\ding{\dingNumber}] Deve ser acionado o botão de comando do deslocamento que está situada nas Respectivas Unidades conforme
necessidade;
    \item[\ding{\dingNumber}] Verificar entre as unidades se não existem pessoas, peças, etc, em seu interior ou em frente antes de qualquer
deslocamento das unidades;
    \item[\ding{\dingNumber}] Nunca faça deslocamentos das unidades manualmente, ou seja, empurrando as unidades. O sistema de
segurança não foi projetado para travar a máquina nestas situações.

\end{generalSafetyToMoveUnities}

\subsection{Limpeza de clichês}

Para garantir a segurança, é crucial seguir algumas precauções ao limpar os clichês:

\newlist{generalRulesToPlateCleaning}{itemize}{1}
\setlist[generalRulesToPlateCleaning]{label ={},nosep, before=\vspace{10pt}, after=\vspace{10pt}}
\begin{generalRulesToPlateCleaning}
\item[\ding{\dingNumber}] Utilizar os procedimentos de segurança para desabilitar as unidades para ter acesso interno para troca de
clichês nas impressoras e forma na unidade perfuradora;
\item[\ding{\dingNumber}] Ao substituir os clichês, formas ou mantas, não apoiar-se nos rolos, preferencialmente deve ser utilizadas as
duas mãos para executar esta tarefa e utilizar os pedais/botões para movimentar os rolos para frente ou para traz
conforme necessidade;
\item[\ding{\dingNumber}] Para efetuar limpeza dos clichês, não apoiar-se nos rolos, preferencialmente devem ser utilizadas as duas mãos
para executar esta tarefa e utilizar os pedais/botões para movimentar os rolos para frente ou para traz conforme
necessidade;
\end{generalRulesToPlateCleaning}

\newpage
\thispagestyle{fancy}
\vspace*{40 pt}

\subsection{Limpeza e manutenção}

A limpeza e manutenção regular do equipamento é crucial para garantir sua eficiência e prolongar sua vida útil. É importante seguir as instruções de limpeza e manutenção e realizar essas tarefas com cuidado e precaução:

\newlist{generalRulesForCleaning}{itemize}{1}
\setlist[generalRulesForCleaning]{label ={},nosep, before=\vspace{10pt}, after=\vspace{10pt}}
\begin{generalRulesForCleaning}
\item[\ding{\dingNumber}]  Durante a limpeza e manutenção a máquina normalmente deve ser travada o acionamento por meio da chave
de emergência. Se necessário for adentrar em pontos de alto risco (embaixo ou sobre o elevador, em frente as
lanças de separação de caixas, etc) o recomendado é desenergizar os painéis para evitar movimentos indesejados
na máquina que podem provocar acidentes;
\item[\ding{\dingNumber}] Se os ajustes elétricos da máquina são usados, deve-se tomar precauções especiais como por exemplo uma
pessoa ficar responsável a supervisionar e impedir a ação de outra pessoa a uma operação perigosa;
\item[\ding{\dingNumber}] Após o término da limpeza ou manutenção as tampas, proteções e carenagens devem ser colocadas em seus
lugares;
\item[\ding{\dingNumber}] A manutenção regular da máquina deve incluir verificações dos dispositivos de segurança, sensores de
segurança e chaves de emergência;
\item[\ding{\dingNumber}] Modificações que irão influenciar no funcionamento dos sistemas de segurança nunca devem ser feitas. As
proteções, carenagens, etc, nunca devem ser eliminadas;

\end{generalRulesForCleaning}

